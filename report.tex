\documentclass[a4j, uplatex, fleqn, dvipdfmx]{jsarticle} % fleqnで数式を左に寄せる,中央揃えにしたい数式は「ceqn環境(nccmath)」に入れればよい
% \documentclass[a4j, uplatex, twocolumn, fleqn, dvipdfmx]{jsarticle} % fleqnで数式を左に寄せる,中央揃えにしたい数式は「ceqn環境(nccmath)」に入れればよい
\usepackage[dvipdfmx]{graphicx, color}
\usepackage{bm} % 太字
\usepackage{amsmath,amssymb} % 行列など
\usepackage{here} % 画像を入れたい場所に入れられる
\usepackage{fancybox,ascmac} % itembox用
\usepackage{hhline} % 表組の2重線を思い通りにできる
\usepackage{siunitx} % SI単位系
\usepackage{subcaption}

\usepackage[dvipdfmx, hidelinks]{hyperref} % hidelinksでリンクに色や枠が付かないようにする
\usepackage{pxjahyper}
\hypersetup{% hyperrefオプションリスト
  setpagesize=false, % 不具合回避
  bookmarksnumbered=true, % ブックマークに番号を付ける
  bookmarksopen=true, % pdfのブックマークのツリーをデフォルトで開く(コンパイル環境によっては機能しないらしい)
  colorlinks=true,
  linkcolor=blue,
  citecolor=blue,
  urlcolor=cyan
}

% きれいな丸囲み数字
\usepackage{pifont}

%余白は上下30mm、左右20mm
\setlength{\textheight}{\paperheight}
\setlength{\topmargin}{4.6truemm}
\addtolength{\topmargin}{-\headheight}
\addtolength{\topmargin}{-\headsep}
\addtolength{\textheight}{-60truemm}

\setlength{\textwidth}{\paperwidth}
\setlength{\oddsidemargin}{-5.4truemm}
\setlength{\evensidemargin}{-5.4truemm}
\addtolength{\textwidth}{-40truemm}

\title{\vspace{-1.5cm}オペレーションズ・リサーチ特論\\レポート2}
% \title{汎用的学習ゲームプラットフォームの作成}
\author{宮崎大学 工学研究科 工学専攻 情報\\T2103329 東郷 拓弥}
% \date{} % タイトルの日付を非表示にする
\begin{document}
\maketitle

\section{設問}
Let $\bm{S} = \{\bm{abb}, \bm{bba}, \bm{bbc}, \bm{bbb}\}$.
\subsection{問題1}
Find a shortest superstring of $\bm{S}$.
\subsubsection{回答}
\[
  \bm{s_1} = \bm{abb}, \bm{s_2} = \bm{bba}, \bm{s_3} = \bm{bbc}, \bm{s_4} = \bm{bbb}
\]
とおく。

全$4!=24$通りのパターンについて考え、文字数を数える。
\begin{align*}
  \bm{s_1} + \bm{s_2} + \bm{s_3} + \bm{s_4} &= \bm{abbabbcbbb} & (10\mbox{文字})\\
  \bm{s_1} + \bm{s_2} + \bm{s_4} + \bm{s_3} &= \bm{abbabbbc} & (8\mbox{文字})\\
  \bm{s_1} + \bm{s_3} + \bm{s_2} + \bm{s_4} &= \bm{abbcbbabbb} & (10\mbox{文字})\\
  \bm{s_1} + \bm{s_3} + \bm{s_4} + \bm{s_2} &= \bm{abbcbbba} & (8\mbox{文字})\\
  \bm{s_1} + \bm{s_4} + \bm{s_2} + \bm{s_3} &= \bm{abbbabbc} & (8\mbox{文字})\\
  \bm{s_1} + \bm{s_4} + \bm{s_3} + \bm{s_2} &= \bm{abbbcbba} & (8\mbox{文字})\\
  \bm{s_2} + \bm{s_1} + \bm{s_3} + \bm{s_4} &= \bm{bbabbcbbb} & (9\mbox{文字})\\
  \bm{s_2} + \bm{s_1} + \bm{s_4} + \bm{s_3} &= \bm{bbabbbc} & (7\mbox{文字})\\
  \bm{s_2} + \bm{s_3} + \bm{s_1} + \bm{s_4} &= \bm{bbabbcabbb} & (10\mbox{文字})\\
  \bm{s_2} + \bm{s_3} + \bm{s_4} + \bm{s_1} &= \bm{bbabbcbbbabb} & (12\mbox{文字})\\
  \bm{s_2} + \bm{s_4} + \bm{s_1} + \bm{s_3} &= \bm{bbabbbabbc} & (10\mbox{文字})\\
  \bm{s_2} + \bm{s_4} + \bm{s_3} + \bm{s_1} &= \bm{bbabbbcabb} & (10\mbox{文字})\\
  \bm{s_3} + \bm{s_1} + \bm{s_2} + \bm{s_4} &= \bm{bbcabbabbb} & (10\mbox{文字})\\
  \bm{s_3} + \bm{s_1} + \bm{s_4} + \bm{s_2} &= \bm{bbcabbba} & (8\mbox{文字})\\
  \bm{s_3} + \bm{s_2} + \bm{s_1} + \bm{s_4} &= \bm{bbcbbabbb} & (9\mbox{文字})\\
  \bm{s_3} + \bm{s_2} + \bm{s_4} + \bm{s_1} &= \bm{bbcbbabbbabb} & (12\mbox{文字})\\
  \bm{s_3} + \bm{s_4} + \bm{s_1} + \bm{s_2} &= \bm{bbcbbbabba} & (10\mbox{文字})\\
  \bm{s_3} + \bm{s_4} + \bm{s_2} + \bm{s_1} &= \bm{bbcbbbabb} & (9\mbox{文字})\\
  \bm{s_4} + \bm{s_1} + \bm{s_2} + \bm{s_3} &= \bm{bbbabbabbc} & (10\mbox{文字})\\
  \bm{s_4} + \bm{s_1} + \bm{s_3} + \bm{s_2} &= \bm{bbbabbcbba} & (10\mbox{文字})\\
  \bm{s_4} + \bm{s_2} + \bm{s_1} + \bm{s_3} &= \bm{bbbabbc} & (7\mbox{文字})\\
  \bm{s_4} + \bm{s_2} + \bm{s_3} + \bm{s_1} &= \bm{bbbabbcabb} & (10\mbox{文字})\\
  \bm{s_4} + \bm{s_3} + \bm{s_1} + \bm{s_2} &= \bm{bbbcabba} & (8\mbox{文字})\\
  \bm{s_4} + \bm{s_3} + \bm{s_2} + \bm{s_1} &= \bm{bbbcbbabb} & (9\mbox{文字})
\end{align*}
よって、最短は7文字で、
$\bm{s_2} + \bm{s_1} + \bm{s_4} + \bm{s_3} = \bm{bbabbbc}$または
$\bm{s_4} + \bm{s_2} + \bm{s_1} + \bm{s_3} = \bm{bbbabbc}$

\subsection{問題2}
Find an approximate solution of the superstring for $\bm{S}$
by applying "Greedy set cover algorithm".
\subsubsection{回答}

% 画像
% \begin{figure}[htbp]
%   \centering
%   \includegraphics[width=7cm]{images/XXX.png}
%   \caption{XXX}
%   \label{fig:XXX}
% \end{figure}

% 画像(横に二枚)
% \begin{figure}[htbp]
%   \centering
%   \begin{tabular}{c}
%     \begin{minipage}{0.45\hsize}
%       \centering
%       \includegraphics[width=7cm]{img/XXX.pdf}
%       \caption{XXX}
%       \label{fig:XXX}
%     \end{minipage}

%     \begin{minipage}{0.45\hsize}
%     \centering
%     \includegraphics[width=7cm]{img/XXX.pdf}
%     \caption{XXX}
%     \label{fig:XXX2}
%     \end{minipage}
% \end{tabular}
% \end{figure}

% 表
% \begin{table}[htb]
%   \caption{XXX}
%   \label{table:XXX}
%   \centering
%   \begin{tabular}{crrr}
%     \hline
%     画像 & 平均画素値(r) & 平均画素値(g) & 平均画素値(b) \\
%     \hline \hline
%     1番目 & 84.557053 & 102.190216 & 107.264511 \\
%     2番目 & 100.516571 & 126.928757 & 138.014618 \\
%     \hline
%   \end{tabular}
% \end{table}

% 画像 マトリックス状 参考:http://www.yamamo10.jp/~yamamoto/comp/latex/make_doc/insert_fig/index.php
% \begin{figure}[htbp]
%   \begin{minipage}[b]{0.32\linewidth}
%     \centering
%     \includegraphics[keepaspectratio, scale=0.3]{images/Q3/xor_sig02.png}
%     \subcaption{$\sigma=0.2$}
%   \end{minipage}
%   \begin{minipage}[b]{0.32\linewidth}
%     \centering
%     \includegraphics[keepaspectratio, scale=0.3]{images/Q3/xor_sig02.png}
%     \subcaption{$\sigma=0.3$}
%   \end{minipage}
%   \begin{minipage}[b]{0.32\linewidth}
%     \centering
%     \includegraphics[keepaspectratio, scale=0.3]{images/Q3/xor_sig05.png}
%     \subcaption{$\sigma=0.5$}
%   \end{minipage}
%   \\
%   \begin{minipage}[b]{0.32\linewidth}
%     \centering
%     \includegraphics[keepaspectratio, scale=0.3]{images/Q3/xor_sig1.png}
%     \subcaption{$\sigma=1$}
%   \end{minipage}
%   \begin{minipage}[b]{0.32\linewidth}
%     \centering
%     \includegraphics[keepaspectratio, scale=0.3]{images/Q3/xor_sig2.png}
%     \subcaption{$\sigma=2$}
%   \end{minipage}
%   \begin{minipage}[b]{0.32\linewidth}
%     \centering
%     \includegraphics[keepaspectratio, scale=0.3]{images/Q3/xor_sig3.png}
%     \subcaption{$\sigma=3$}
%   \end{minipage}
%   \caption{XOR問題}
%   \label{fig:q3_xor}
% \end{figure}

% 画像 表形式
% \begin{table}[H]
%   \caption{入力画像1-inp.pgm, 1-tpl.pgmと出力画像}
%   \begin{tabular}{|c|c|}
%     \hline 
%     入力画像 & テンプレート画像
%     \\ \hline
%     \begin{minipage}[]{0.49\linewidth}
%       \centering
%       \includegraphics[keepaspectratio, scale=0.3]{images/input/1-inp.png}
%       % \subcaption{1-inp.pgm}
%     \end{minipage}
%     &
%     \begin{minipage}[]{0.49\linewidth}
%       \centering
%       \mbox{\raisebox{0mm}{\includegraphics[keepaspectratio, scale=0.3]{images/input/1-tpl.png}}}
%       % \subcaption{1-tpl.pgm}
%     \end{minipage}
%     \\ \hline
%     \hline
%     出力画像 & チェック画像
%     \\ \hline
%     \begin{minipage}[]{0.49\linewidth}
%       \centering
%       \includegraphics[keepaspectratio, scale=0.3]{images/output/1-out.png}
%       % \subcaption{1-out.pgm}
%     \end{minipage}
%     &
%     \begin{minipage}[]{0.49\linewidth}
%       \centering
%       \includegraphics[keepaspectratio, scale=0.3]{images/output/1-chk.png}
%       % \subcaption{1-chk.pgm}
%     \end{minipage}
%     \\ \hline
%   \end{tabular}
%   \label{table:1-inp}
% \end{table}

\end{document}
